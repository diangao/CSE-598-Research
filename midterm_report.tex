\section{Introduction}
\subsection{Background and Motivation}
The effectiveness of state representation learning constitutes a fundamental challenge in developing robust multi-agent systems, particularly for applications requiring long-term interaction and cross-environment adaptability. While current LLM-based agents demonstrate proficiency in isolated decision-making tasks, three critical gaps persist:

\begin{itemize}
    \item \textbf{Representation-Environment Mismatch}: Fixed memory architectures struggle to adapt between discrete symbolic reasoning (e.g., game rules) and continuous latent-space decision-making
    
    \item \textbf{Multi-Horizon Coordination}: Existing systems lack mechanisms for maintaining coherent state representations across varying interaction timescales in multi-agent scenarios
    
    \item \textbf{Post-Training Instability}: Learned representations often degrade when deployed in environments differing from their training regimes
\end{itemize}

Our work addresses these challenges through a structured investigation of three state representation paradigms in Cognitive Language Agents:

\begin{itemize}
    \item \textbf{GraphDB}: Explicit relational modeling for strategic game trees
    \item \textbf{VectorDB}: Continuous embedding spaces for probabilistic reasoning
    \item \textbf{Semantic Memory}: Hybrid neuro-symbolic representations via LLM abstraction
\end{itemize}

\subsection{Research Objectives}
Our investigation establishes three principal research objectives that systematically address the core challenges in state representation learning for multi-agent systems:

\begin{enumerate}
    \item \textbf{Architecture-Specific Reasoning Capacity Analysis}
    
    This objective focuses on quantifying how different memory architectures influence strategic decision-making in structured environments. Through controlled multi-agent Tic-Tac-Toe experiments, we evaluate:
    
    \begin{itemize}
        \item Multi-step planning efficiency through $\tau$-bench metrics
        \item Planning horizon supported by different memory architectures
        \item Coordination efficiency differentials through win rate analysis
        \item Effectiveness of memory mechanisms in reducing redundant moves
    \end{itemize}

    \item \textbf{Cross-Environment Representation Transfer Assessment}
    
    This objective examines the adaptability of learned state representations across decision-making regimes. Using our continuous Tic-Tac-Toe variants, we analyze:
    
    \begin{itemize}
        \item Quantitative transferability metrics between discrete and continuous spaces
        \item Strategy consistency through decision distribution analysis
        \item Failure recovery mechanisms in continuous state spaces
    \end{itemize}

    \item \textbf{Post-Training Optimization Framework Validation}
    
    This objective evaluates enhancement strategies for learned representations:
    
    \begin{itemize}
        \item Memory stability across extended interaction horizons
        \item Cross-architecture knowledge transfer effectiveness
        \item Adaptation speed for novel task variations
    \end{itemize}
\end{enumerate}

\subsection{Experimental Design}
Our experimental framework is built upon the $\tau$-bench framework, which provides a standardized environment for evaluating LLM-based agents. The framework consists of:

\begin{itemize}
    \item \textbf{Environment}: A Tic-Tac-Toe implementation that supports both discrete and continuous state representations
    \item \textbf{Memory Architectures}: Three distinct memory implementations:
    \begin{itemize}
        \item GraphDB: Stores game states as nodes with legal move edges
        \item VectorDB: Encodes states as $\mathbb{R}^d$ embeddings
        \item Semantic Memory: Combines symbolic rules with LLM abstractions
    \end{itemize}
    \item \textbf{Evaluation Metrics}: 
    \begin{itemize}
        \item $\tau$-style Task Success Rate: Win percentage across multiple games
        \item State Retrieval Latency: Memory access efficiency
        \item Move Efficiency: Reduction in redundant moves
        \item Generalization Score: Adaptation to novel game variants
    \end{itemize}
\end{itemize}

\subsection{Preliminary Results and Findings}
Our initial experiments with the Tic-Tac-Toe environment have revealed several key insights:

\begin{itemize}
    \item \textbf{Memory Performance}: Different memory architectures show varying levels of effectiveness in terms of:
    \begin{itemize}
        \item Retrieval times (GraphDB typically fastest, Semantic Memory slowest)
        \item Storage efficiency (VectorDB most compact, GraphDB most explicit)
        \item Decision quality (preliminary results suggest GraphDB may have advantages in strategic planning)
    \end{itemize}
    
    \item \textbf{Behavioral Patterns}: Analysis of agent behavior reveals:
    \begin{itemize}
        \item Memory utilization patterns across different game phases
        \item Decision-making strategies influenced by memory architecture
        \item Potential trade-offs between memory efficiency and decision quality
    \end{itemize}
\end{itemize}

These findings provide a foundation for our ongoing investigation into the relationship between memory architecture and agent performance in multi-agent systems. 